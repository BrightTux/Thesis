
The use of video data as a means of surveillance is no longer a new idea. Thousands, if not millions of video footage are recorded by the second. However, there has not been much post-processing done to these semantic rich source of data. In this work, carpark surveillance video data which is rich with semantics is tapped into so as to extract useful data for analytic purposes. Example of such semantics includes vehicle color, trajectories, travelled speed, weather information, types of vehicles and the such. In this work, a vehicle detection and tracking algorithm was adopted and used in conjunction with the proposed large scale semantics extraction module to extract color and motion information from a month's worth of data. 

Subsequently, a semantic retrieval module is proposed and used to search for all relevant video shots based on a given semantic query. This is a challenging task as outdoor scenarios are often non-ideal due to its ever-changing illumination and weather conditions which would directly affect the accuracy of the colors of the vehicles. To address these challenges, we subdivided the source video data into smaller chunks by introducing spatio-temporal cubes or \emph{atoms}, which is used to store and retrieve these semantics at ease. The proposed method was tested on a month of continuous data from an outdoor carpark under various lighting and weather conditions. 

The Recall@k along with the normalized Discounted Cumulative Gain (nDCG) metric was used to measure the overall performance of the proposed retrieval engine. Along with that, the performance of the proposed method was also measured in terms of retrieval speed. From the experimental evaluations, the results shows promising results for the retrieval module in terms of precision, ranking of results as well as retrieval time. As such, the retrieval module could benefit the deep learning research field when used as a semi-supervised ground truth annotation tool.


%\keywords{Vehicle Semantic Extraction, Retrieval Systems, Carpark Surveillance}